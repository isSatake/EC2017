%%
%% 研究報告用スイッチ(情報処理学会用ファイルをEC2017用に変更)
%% [techreq]
%%
%% 欧文表記無しのスイッチ(etitle,jkeyword,eabstract,ekeywordは任意)
%% [noauthor]
%%

\documentclass[submit,techreq]{ec2017}
%\documentclass[submit,techreq,noauthor]{ipsj}


\usepackage[dvips]{graphicx}
\usepackage{latexsym}

\def\Underline{\setbox0\hbox\bgroup\let\\\endUnderline}
\def\endUnderline{\vphantom{y}\egroup\smash{\underline{\box0}}\\}
\def\|{\verb|}

\setcounter{巻数}{57}%vol53=2012
\setcounter{号数}{10}
\setcounter{page}{1}


\begin{document}


\title{Re:Piano: 演奏履歴を活用する楽器演奏支援システム}

\etitle{Re:Piano: Supporting music session with past performance}

\affiliate{政メ}{慶應義塾大学政策・メディア研究科\\
Keio University}
\affiliate{環境}{慶應義塾大学環境情報学部\\
Keio University}

\author{佐竹 紘明}{Satake Hiroaki}{政メ}[stk@sfc.keio.ac.jp]
\author{増井 俊之}{Masui Toshiyuki}{環境}[masui@pitecan.com]

\begin{abstract}
(ここは結論を冒頭に書く)
 演奏履歴をリアルタイムに活用できる楽器及びそれを利用した新しい演奏法を提案する。一般に、録音された演奏データは演奏後に加工されて活用されるものであるが、演奏データを演奏時にリアルタイムに活用できるシステムや楽器はほとんど存在しない。楽器の演奏履歴を常に記録しておき、演奏時にそれをリアルタイムに利用することで多彩な演奏を可能にするシステム「Re:Piano」を試作した。
\end{abstract}


%\begin{jkeyword}
%情報処理学会論文誌ジャーナル,\LaTeX,スタイルファイル,べからず集
%\end{jkeyword}
%
\begin{eabstract}
This document is a guide to prepare a draft for submitting to IPSJ
Journal, and the final camera-ready manuscript of a paper to appear in
IPSJ Journal, using {\LaTeX} and special style files.  Since this
document itself is produced with the style files, it will help you to
refer its source file which is distributed with the style files.
\end{eabstract}

%\begin{ekeyword}
%IPSJ Journal, \LaTeX, style files, ``Dos and Dont's'' list
%\end{ekeyword}


\maketitle

%1
\section{はじめに}
\label{sec:start}

ひとりで楽器を演奏するのは、それほど楽しくないものである。これは楽曲として聞こえる演奏をひとりで行なうのが難しいからである。特に初心者の場合、 自分の下手な演奏を聞くのが悲しいし、満足できるほど上手くなるまで練習するのは大変である。初心者でない場合でも、単音だけ聞いて楽しい曲はほとんど無いことから、単音楽器をひとりで楽しく演奏することは難しい。楽器をひとりで演奏して楽しいのは、熟練者がピアノやギターといった独奏楽器を弾くときぐらいかもしれない。
 
 一方、下手であっても合奏に参加するのは楽しいものである。カラオケは歌が下手でも人数が多ければ盛り上がることができるし、小学校でのリコーダーや鍵盤ハーモニカによる楽しい合奏は、技量に関係なく誰もが体験している。
演奏技術が足りなくても、合奏のように沢山の音を重ねて重厚な音楽を作れれば、誰もが楽器演奏を楽しむことが可能である。初心者や単音楽器奏者でも、合奏的に演奏を楽しむことができれば練習や上達にも効果があると思われる。

 その場限りで楽器演奏を行なうのではなく、演奏履歴を利用して音を重ねたり、連続再生したりして過去の自分と簡単に合奏することができれば、初心者でも、単音楽器でももっと演奏を楽しめる可能性がある。本論文では、こういう思想にもとづいて作成したRe:Pianoを提案する。

%2
\section{Re:Piano}

Re:Pianoは、自分の過去の演奏履歴を活用することによって演奏をより楽しくすることができるシステムである。演奏中に以下のボタンを押すことで楽しい演奏ライフをサポートする。

%2.1
\subsection{録音再生ボタン}
直前の無音部分から現在までの演奏を登録して、繰り返し再生を行う。登録部分の中に演奏の繰り返しが含まれる場合は、繰り返し部分だけを登録して連続再生を行う。 (Dynamic Macro)

登録された部分を再生中に重ねて演奏を行なうことができるが、そこで録音再生ボタンを押すとその演奏も登録される。

%2.2
\subsection{やり直しボタン}
前に戻ってやり直し
    前回の繰り返しに戻る
   何度か押すと完全クリアされる

%3
\section{Re:Piano使用例}

前述のような単純な手法でいろんな楽しみ方ができることを主張

\subsection{すごく下手でも上手く演奏できる例}
  速弾きができない人でもなんとかなる例
  
\subsection{Dynamic Macroの例 (実はこれがメイン)}
  DynamicMacro \cite{Masui:1994:RPK:191666.191722} とは
   テキストエディタ用の繰り返し操作を効率化するシステム
   ユーザの繰り返し操作にもとづいて次の操作を予測し、キーボードマクロのように利用できる
   利用するキーは1つでじゅうぶんですよ
   繰り返し操作の開始と終了を正確に指定する必要がなく、操作中のどこでREPEATを押しても操作が再実行される
   操作を行ったあとで繰り返し実行を指示できる
  実例
\paragraph*{ベースラインの上でインプロビゼーションする}
ああ
\paragraph*{イカしたフレーズに伴奏をつける (?)}
いい

%4
\section{実装}

\subsection{Web技術による実装}
  Re:PianoはJavaScriptとWebMIDIAPIで実装されたブラウザアプリケーションであり、MIDIキーボードを接続することで利用できる
  最新のWebブラウザでは、WebMIDIAPIを利用すれば簡単にMIDI機器と連携したアプリケーションを開発することが可能であり、本論文のような実験も容易である
\subsection{MIDIメッセージの取り扱い}
  MIDIMessageEventというobjectが飛んでくる
  音程、強さ、タイムスタンプといったノート情報が含まれている
  これをブラウザ内音源に渡したり、Garagebandといった外部MIDI音源に送信することで演奏される
\subsection{ループの管理}
  ひとつひとつをスタックに積んでいる
  undoでpopする
  試行錯誤が簡単


%5
\section{議論}

 本当にたーのしーのか?
 Repeanoが適する/適さない利用シーン
  適さないとこなど最後に書けばいい [増井俊之.icon]
\subsection{繰り返し録音ボタンは[* 演奏の後で押せる]のがポイント}
  繰り返しフレーズを録音したい場合
  タマタマ良いフレーズが弾けたときそれを再生できるのは嬉しい
 普通の重ね録音も楽勝で可能
\subsection{既存の手法との比較}
  ひとりで録音した演奏を重ねることはよく行なわれている
   たとえばDTM
  しかし機材の用意や操作は面倒だし大変である
   画面に向かってやる作業
  ループ式のものもあるが、それでも操作は面倒だし制約も多い (要調査)
   録音開始+停止の操作が必要
  	試行錯誤しづらい
    Re:Piano->操作が単純なので試行錯誤しても苦にならない
 カラオケとの比較
  バックが上手 vs バックが存在しない
  みんなで楽しい vs ぼっち
  
%6
\section{結論}

たーのしー!



\bibliographystyle{ipsjsort}

\bibliography{main}


\end{document}
