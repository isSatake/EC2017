%%
%% 研究報告用スイッチ(情報処理学会用ファイルをEC2017用に変更)
%% [techreq]
%%
%% 欧文表記無しのスイッチ(etitle,jkeyword,eabstract,ekeywordは任意)
%% [noauthor]
%%

\documentclass[submit,techreq]{ec2017}
%\documentclass[submit,techreq,noauthor]{ipsj}


\usepackage[dvips]{graphicx}
\usepackage{latexsym}

\def\Underline{\setbox0\hbox\bgroup\let\\\endUnderline}
\def\endUnderline{\vphantom{y}\egroup\smash{\underline{\box0}}\\}
\def\|{\verb|}

\setcounter{巻数}{57}%vol53=2012
\setcounter{号数}{10}
\setcounter{page}{1}


\begin{document}


\title{Re:Piano: 演奏履歴を活用する楽器演奏支援システム}

\etitle{Re:Piano: Supporting music session with past performance}

\affiliate{政メ}{慶應義塾大学政策・メディア研究科\\
Keio University}
\affiliate{環境}{慶應義塾大学環境情報学部\\
Keio University}

\author{佐竹 紘明}{Satake Hiroaki}{政メ}[stk@sfc.keio.ac.jp]
\author{増井 俊之}{Masui Toshiyuki}{環境}[masui@pitecan.com]

\begin{abstract}
(ここは結論を冒頭に書く)
演奏履歴をリアルタイムに活用できる楽器及びそれを利用した新しい演奏法を提案する。一般に、録音された演奏データは演奏後に加工されて活用されるものであるが、演奏データを演奏時にリアルタイムに活用できるシステムや楽器はほとんど存在しない。楽器の演奏履歴を常に記録しておき、演奏時にそれをリアルタイムに利用することで多彩な演奏を可能にするシステム「Re:Piano」を試作した。
\end{abstract}


%\begin{jkeyword}
%情報処理学会論文誌ジャーナル,\LaTeX,スタイルファイル,べからず集
%\end{jkeyword}
%
\begin{eabstract}
This document is a guide to prepare a draft for submitting to IPSJ
Journal, and the final camera-ready manuscript of a paper to appear in
IPSJ Journal, using {\LaTeX} and special style files.  Since this
document itself is produced with the style files, it will help you to
refer its source file which is distributed with the style files.
\end{eabstract}

%\begin{ekeyword}
%IPSJ Journal, \LaTeX, style files, ``Dos and Dont's'' list
%\end{ekeyword}


\maketitle

%1
\section{はじめに}
\label{sec:start}

ひとりで楽器を演奏するのはそれほど楽しくないものである。
これは楽曲として聞こえる演奏をひとりで行なうのが難しいからである。
特に初心者の場合、 自分の下手な演奏を聞くのが悲しいし、満足できるほど上手くなるまで練習するのは大変である。
初心者でない場合でも、単音だけ聞いて楽しい曲はほとんど無いことから、単音楽器をひとりで楽しく演奏することは難しい。
楽器をひとりで演奏して楽しいのは、熟練者がピアノやギターといった独奏楽器を弾くときぐらいかもしれない。

一方、下手であっても合奏に参加するのは楽しいものである。
カラオケは歌が下手でも人数が多ければ盛り上がることができるし、小学校でのリコーダーや鍵盤ハーモニカによる楽しい合奏は、技量に関係なく誰もが体験している。
演奏技術が足りなくても、合奏のように沢山の音を重ねて重厚な音楽を作れれば、誰もが楽器演奏を楽しむことが可能である。
初心者や単音楽器奏者でも、合奏的に演奏を楽しむことができれば練習や上達にも効果があると思われる。

その場限りで楽器演奏を行なうのではなく、演奏履歴を利用して音を重ねたり、連続再生したりして過去の自分と簡単に合奏することができれば、
初心者でも、単音楽器でももっと演奏を楽しめる可能性がある。
本論文では、こういう思想にもとづいて作成したRe:Pianoを提案する。

%2
\section{Re:Piano}
\label{repiano}

Re:Pianoは、自分の過去の演奏履歴を活用することによって演奏をより楽しくすることができるシステムである。
演奏中に以下のボタンを押すことで楽しい演奏ライフをサポートする。

%2.1
\subsection{録音再生ボタン}
\label{recplaybutton}
直前の無音部分から現在までの演奏を登録して、繰り返し再生を行う。
登録部分の中に演奏の繰り返しが含まれる場合は、繰り返し部分だけを登録して連続再生を行う。

再生中に重ねて演奏を行なうことができるが、そこで録音再生ボタンを押すとその演奏も新たに登録される。

%2.2
\subsection{やり直しボタン}
\ref{recplaybutton}で登録された演奏を、新しいものから順に取り消す。

%3
\section{Re:Piano使用例}

\ref{repiano}で述べた単純な手法によって、さまざまな演奏の楽しみ方が可能になる。

\subsection{すごく下手でも上手く演奏できる例}
速弾きができない人でもなんとかなる例
一つ一つのフレーズを重ねていくことで完成度の高い演奏にしていく


\subsection{Dynamic Macroの例 (実はこれがメイン)}
DynamicMacro\cite{Masui}とは
テキストエディタ用の繰り返し操作を効率化するシステム
ユーザの繰り返し操作にもとづいて次の操作を予測し、キーボードマクロのように利用できる
利用するキーは1つでじゅうぶんですよ
繰り返し操作の開始と終了を正確に指定する必要がなく、操作中のどこでREPEATを押しても操作が再実行される
操作を行ったあとで繰り返し実行を指示できる
実例
\paragraph*{ベースラインの上でインプロビゼーションする}
ああ
\paragraph*{イカしたフレーズに伴奏をつける (?)}
いい

%4
\section{実装}

\subsection{Web技術による実装}
Re:PianoはJavaScriptとWeb MIDI API\cite{webmidi}で実装されたWebブラウザアプリケーションであり、MIDIキーボードを接続することで利用できる。
最新のWebブラウザでは、Web MIDI APIを利用することで簡単にMIDI機器と連携したアプリケーションを開発することが可能であり、本論文のような実験も容易である。
また、Web Audio API\cite{webaudio} を利用して音源をブラウザ内に埋め込んでおり、外部MIDI音源を利用することなく演奏を楽しむことができる。

%5
\section{議論}

本当にたーのしーのか?
Repeanoが適する/適さない利用シーン
\subsection{繰り返し録音ボタンは演奏の後で押せるのがポイント}
繰り返しフレーズを録音したい場合
タマタマ良いフレーズが弾けたときそれを再生できるのは嬉しい
普通の重ね録音も楽勝で可能

\subsection{既存の手法との比較}

%hogeをするのに、既存のfugaという手法よりも本論文の手法の方がこういう点で優れていたり、こういう制約がある、ということを述べる
%hoge -> 演奏履歴を活用する楽器演奏
%fuga -> DTMなどにおける多重録音、サンプラー、ルーパー

DTM
\footnote{\textsf{DeskTop Music\\パソコンと電子楽器などを接続し、専用のソフトウェアで演奏したり音楽制作を行う行為}}
では、ひとりで録音した演奏を重ねていく多重録音はポピュラーな手法としてよく行なわれている。
しかし、機材を用意したり、専用ソフトウェアの操作を覚えるのは大変である。
楽曲としての完成度が求められる利用シーンには適するが、演奏を行いながらリアルタイムに演奏履歴を活用することは難しい。

サンプラーを利用すると、音声素材や演奏を登録し、それらを自由に呼び出して演奏に利用できる。
これは事前に素材を用意しておくことが前提となっており、それなりの準備が必要である。
こういったことから、演奏中に偶然思いついたフレーズを記録するのも難しい。

常に演奏を記録しておくことで、良いと感じたフレーズをすぐに再利用できるのは本システムの重要なポイントである。
良いフレーズを忘れてしまうことを防げるし、すぐ演奏に活用できることでより演奏を楽しむことができる。

そのときの演奏内容に応じて録音・再生する音声を簡単に変更できることで、Re:Pianoの利用可能なシーンは大きく広がる。
セッションの雰囲気に合わせて再生する音声を変えたり

単純な操作だから、試行錯誤しやすい

演奏履歴を活用する楽器演奏
孤独な日常の練習でも気軽に合奏的な演奏を楽しむことができる

%試行錯誤しづらい
%これまた専用の機材が必要
%でもRe:Pianoにもブラウザが必要→結局一緒でわ

ルーパーを利用すると、演奏を繰り返し再生してひとりでも合奏的な演奏を楽しむことができる。
しかし、明確な録音開始・停止の操作が必要であり、タイミングを合わせるのが大変である。
また常に録音しておき、演奏に活用するという用途には利用できない。
試行錯誤しづらい


Re:Piano->操作が単純なので試行錯誤しても苦にならない
Dynamic Macroが使えるので、正確に録音開始・停止位置を指定する必要がない

カラオケとの比較
バックが上手 vs バックが存在しない
みんなで楽しい vs ぼっち

%6
\section{結論}

たーのしー!



\bibliographystyle{ipsjsort}

\bibliography{main}


\end{document}
