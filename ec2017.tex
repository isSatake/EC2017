%%
%% 研究報告用スイッチ(情報処理学会用ファイルをEC2017用に変更)
%% [techreq]
%%
%% 欧文表記無しのスイッチ(etitle,jkeyword,eabstract,ekeywordは任意)
%% [noauthor]
%%

\documentclass[submit,techreq]{ec2017}
%\documentclass[submit,techreq,noauthor]{ipsj}


\usepackage[dvips]{graphicx}
\usepackage{latexsym}

\def\Underline{\setbox0\hbox\bgroup\let\\\endUnderline}
\def\endUnderline{\vphantom{y}\egroup\smash{\underline{\box0}}\\}
\def\|{\verb|}

\setcounter{巻数}{57}%vol53=2012
\setcounter{号数}{10}
\setcounter{page}{1}


\begin{document}


\title{Re:Piano: 演奏履歴を活用する楽器演奏支援システム}

\etitle{Re:Piano: Supporting music session with past performance}

\affiliate{政メ}{慶應義塾大学政策・メディア研究科\\
Keio University}
\affiliate{環境}{慶應義塾大学環境情報学部\\
Keio University}

\author{佐竹 紘明}{Satake Hiroaki}{政メ}[stk@sfc.keio.ac.jp]
\author{増井 俊之}{Masui Toshiyuki}{環境}[masui@pitecan.com]

\begin{abstract}
(ここは結論を冒頭に書く)
演奏履歴をリアルタイムに活用できる楽器及びそれを利用した新しい演奏法を提案する。一般に、録音された演奏データは演奏後に加工されて活用されるものであるが、演奏データを演奏時にリアルタイムに活用できるシステムや楽器はほとんど存在しない。楽器の演奏履歴を常に記録しておき、演奏時にそれをリアルタイムに利用することで多彩な演奏を可能にするシステム「Re:Piano」を試作した。
\end{abstract}

%\begin{jkeyword}
%情報処理学会論文誌ジャーナル,\LaTeX,スタイルファイル,べからず集
%\end{jkeyword}
%
\begin{eabstract}
This document is a guide to prepare a draft for submitting to IPSJ
Journal, and the final camera-ready manuscript of a paper to appear in
IPSJ Journal, using {\LaTeX} and special style files.  Since this
document itself is produced with the style files, it will help you to
refer its source file which is distributed with the style files.
\end{eabstract}

%\begin{ekeyword}
%IPSJ Journal, \LaTeX, style files, ``Dos and Dont's'' list
%\end{ekeyword}

\maketitle

%
% 1
%
\section{はじめに}
\label{sec:start}

ひとりで楽器を演奏するのはそれほど楽しくないものであるが、
これは楽曲として聞こえる演奏をひとりで行なうのが難しいという理由が大きいと思われる。
%
初心者の場合、 自分の下手な演奏を聞くのは悲しいし、満足できるほど上手くなるまで練習するのは大変である。
初心者でない場合でも、単音だけ聞いて楽しい曲はほとんど無いため、単音楽器をひとりで楽しく演奏することは難しい。
楽器をひとりで演奏して楽しいのは、熟練者がピアノやギターといった独奏楽器を弾くときぐらいかもしれない。
% 独奏楽器ていうの? (masui)

一方、下手であっても合奏に参加するのは楽しいものである。
カラオケは歌が下手でも
人数が多ければ % ??? 伴奏があるからでは? (masui)
盛り上がることができるし、
小学校でのリコーダーや鍵盤ハーモニカによる合奏は
多くの人にとって楽しい思い出として記憶されていると思われる。
% 楽しい合奏は、技量に関係なく誰もが体験している。
演奏技術が足りなくても、合奏のように沢山の音を重ねて重厚な音楽を作れれば、誰もが楽器演奏を楽しむことが可能である。
初心者や単音楽器奏者でも、合奏的に演奏を楽しむことができれば練習や上達にも効果があると思われる。
%
その場限りで楽器演奏を行なうのではなく、
演奏履歴を利用して音を重ねたり、
連続再生したりして過去の自分と簡単に合奏することができれば、
初心者でも、単音楽器でももっと演奏を楽しめる可能性がある。

ひとりでも合奏的な演奏を可能にするツールは既に存在している。
%
ひとりで録音した演奏を重ねていく多重録音
(DTM: Desktop Music\footnote{パソコンと楽器やマイクなどを接続し、
    専用のソフトウェアで演奏したり音楽制作を行う行為})が
ポピュラーになっているが\cite{jacob}\cite{resound}、
DTMを楽しむには相当な準備や作業が必要なので
カジュアルに合奏を楽しむことは難しい。
%
% これがDTMの例というのはよくわからない (増井)
% DTMというより多重録音の例ですねぇ (satake)
%
また、サンプラー\footnote{音声を登録し、
    ボタンなどのインターフェースを利用してそれらを自由に再生できる装置}
やルーパー\footnote{音声を重ねて記録し、ループ再生する装置}
といったツールを利用すると音声素材や演奏を登録し、それらを自由に呼び出して演奏に活用したり、
繰り返し再生して伴奏をさせることが可能である。

これらのツールはあらかじめ演奏全体の完成図が見えており、
レコーディングやライブパフォーマンスといった高い完成度を求められる利用シーンでは威力を発揮するが、
明確な録音開始・停止の操作が必要であり、なおかつタイミングの正確さが求められるため、
日常的な楽器演奏において何気なく利用するにはハードルが高い。

誰もが日常的に合奏的な演奏を楽しむためにはこのような制約を解決し、
より簡単に演奏履歴を活用できるシステムが必要である。
本論文では、このような思想にもとづいて作成した{\system}を提案する。

%2
\section{\system}
\label{repiano}

%
%{\system}では常に演奏を記録しておくことで、演奏が終わった後からでも素材として登録することが可能である。
%また、Dynamic Macroを利用することで繰り返しフレーズの開始・終了位置を意識しなくても
%演奏を登録することが可能になっている。

{\system}は、自分の過去の演奏履歴を活用することによって演奏をより楽しくすることができるシステムである。
演奏中に以下のボタンを押すことで楽しい演奏ライフをサポートする。

%2.1
\subsection{録音再生ボタン}
\label{recplaybutton}
(図入れたい)

直前の無音部分から現在までの演奏を登録して、繰り返し再生を行う。
録音開始の操作は不要で、演奏の途中や、
演奏が終わってから登録可能な点が既存のツールとは異なる大きな特徴である。
登録部分の中に演奏の繰り返しが含まれる場合はDynamic Macro\cite{masui}を適用し、
繰り返し部分だけを登録して連続再生を行う。
また、再生中に重ねて演奏を行なうことができ、そこで録音再生ボタンを押すとその演奏も新たに登録される。
この演奏は最初に登録された繰り返しフレーズのタイミングに合わせて記録されるため、
時間が経過してもずれることなく再生され続ける。

%2.2
\subsection{やり直しボタン}
(図入れたい)

\ref{recplaybutton}で重ねていった演奏を、新しいものから順に取り消す。
この2つのボタンによって、何気なく弾いたフレーズを重ねていったり、
試行錯誤をしながら合奏的演奏を楽しむことができる。

%3
\section{Re:Piano使用例}

\ref{repiano}で述べた単純な手法によって、さまざまな演奏の楽しみ方が可能になる。

\subsection{すごく下手でも上手く演奏できる例}
速弾きができない人でもなんとかなる例
一つ一つのフレーズを重ねていくことで完成度の高い演奏にしていく


\subsection{Dynamic Macroの例 (実はこれがメイン)}
DynamicMacro\cite{Masui}とは
テキストエディタ用の繰り返し操作を効率化するシステム
ユーザの繰り返し操作にもとづいて次の操作を予測し、キーボードマクロのように利用できる
利用するキーは1つでじゅうぶんですよ
繰り返し操作の開始と終了を正確に指定する必要がなく、操作中のどこでREPEATを押しても操作が再実行される
操作を行ったあとで繰り返し実行を指示できる
実例
\paragraph*{ベースラインの上でインプロビゼーションする}
ああ
\paragraph*{イカしたフレーズに伴奏をつける (?)}
いい

%4
\section{実装}

\subsection{Web技術による実装}
Re:PianoはJavaScriptとWeb MIDI API\cite{webmidi}で実装されたWebブラウザアプリケーションであり、MIDIキーボードを接続することで利用できる。
最新のWebブラウザでは、Web MIDI APIを利用することで簡単にMIDI機器と連携したアプリケーションを開発することが可能であり、本論文のような実験も容易である。
また、Web Audio API\cite{webaudio} を利用して音源をブラウザ内に埋め込んでおり、外部MIDI音源を利用することなく演奏を楽しむことができる。

%5
\section{議論}

本当にたーのしーのか?
Repeanoが適する/適さない利用シーン

\subsection{繰り返し録音ボタンは演奏の後で押せるのがポイント}

繰り返しフレーズを録音したい場合
タマタマ良いフレーズが弾けたときそれを再生できるのは嬉しい
普通の重ね録音も楽勝で可能

\subsection{既存の手法との比較}
サンプラーはどんな音声でも登録できることから、単一の楽器だけでなく、
バリエーション豊かな音源によって多彩な演奏を可能にする強力なツールである。
{\system}では他の楽器の演奏履歴を呼び出すことはできないが、
ひとりでも気軽に合奏的演奏を楽しめることを目標としており、
音声の登録といった事前の準備は必ずしも必要ではなく、
{\system}を用意するだけですぐに利用できる。


\subsection{評価}

著者らが{\system}の試作品を約2ヶ月に渡って使用し、
得られた結果および評価について述べる。

第一著者はクラシックギターに習熟しているので、
ピアノでもある程度楽譜を見て演奏したり、
耳コピ\footnote{\textsf{音楽を耳で聞いて演奏を再現したり、楽譜に起こすこと}}
をすることはできるが、両手を使った演奏をすることはできない。
単音による自身のピアノ演奏をいつも退屈だと感じていたが、
{\system}を日常的に使い始めてからは、そのような意識が緩和された。
繰り返し再生の特性上、技巧的なコードチェンジをしながら演奏を展開していくことは難しいが、
ベースラインやコードを記録してからメロディラインを自由に演奏するスタイルなら
1つ1つのパートに集中でき、なおかつリズムも乱れづらく初心者にとってハードルが低い。
これによって、手軽に合奏的演奏を楽しむことができた。
現状より少しでもレベルの高い演奏を行えるようになることで、
継続的な楽器演奏へのモチベーションが高まった。

第二著者はジャズピアノの演奏を趣味としている。
ピアノでジャズを演奏する場合、ベース/バッキング/メロディを同時に弾きれば嬉しいものであるが、
二本の手でこれらを演奏することは難しいためどこかに負担がかかってしまう。
{\system}を利用して最初にベースパートを録音しておいてから
両手でピアノを重ねて演奏することにより、
簡単にジャズ演奏を楽しめるようになった。

%6
\section{結論}

たーのしー!

今後は、MIDIを用いない生楽器でも{\system}のような演奏を実現する方法について検討していく。


\bibliographystyle{ipsjsort}
\bibliography{main}

\end{document}
