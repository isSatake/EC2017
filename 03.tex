%3
\section{Re:Piano使用例}

\ref{repiano}で述べた単純な手法によって、さまざまな演奏の楽しみ方が可能になる。

\subsection{すごく下手でも上手く演奏できる例}
速弾きができない人でもなんとかなる例
一つ一つのフレーズを重ねていくことで完成度の高い演奏にしていく


\subsection{Dynamic Macroの例 (実はこれがメイン)}
DynamicMacro\cite{Masui}とは
テキストエディタ用の繰り返し操作を効率化するシステム
ユーザの繰り返し操作にもとづいて次の操作を予測し、キーボードマクロのように利用できる
利用するキーは1つでじゅうぶんですよ
繰り返し操作の開始と終了を正確に指定する必要がなく、操作中のどこでREPEATを押しても操作が再実行される
操作を行ったあとで繰り返し実行を指示できる
実例
\paragraph*{ベースラインの上でインプロビゼーションする}
ああ
\paragraph*{イカしたフレーズに伴奏をつける (?)}
いい
