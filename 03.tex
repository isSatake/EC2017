%3
\section{{\system}使用例}

\ref{repiano}で述べた単純な操作によって、
何気なく弾いたフレーズを重ねていったり、
試行錯誤をしながら合奏的演奏を楽しむことができる。

{\system}を利用したさまざまな演奏の楽しみ方について解説する。

% subsection3つくらいほしい

\subsection{初心者でも上手く演奏できる例}
ピアノ初心者にとって、両手を器用にコントロールするのは難しい。
したがって、伴奏とメロディラインを同時に演奏するのは大きなハードルになる。
特に独奏楽器では、伴奏とメロディラインを同時に演奏しなければならないハードルが存在する。
%  初心者にはこういう問題がある
%  両手が使えなくても、一つ一つのフレーズを重ねていくことで完成度の高い演奏にしていく
%  リズムやタイミングが適当でもズレていかないし、試行錯誤すればいい

% 速弾きができない人でもなんとかなる例(初心者?)
%  メロディや伴奏と同時に弾くのは無理でも、細かいフレーズをだけなら弾ける

% ルーパー的機能を意識しなくても使えちゃう例
%  日常的な利用シーンで、何気なく弾いたフレーズにオッいいなと感じて、それに伴奏を付けたり伴奏してもらったりして演奏する

\subsection{Dynamic Macroを利用した〜}
Dynamic Macroとは、入力の繰り返しを自動化するEmacs拡張である。
連続した操作の開始と終了のタイミングを正確に指定する必要がなく、
実際の操作を行ったあとから繰り返し実行を指示できる。

{\system}では、演奏の登録時にこのDynamic Macroを適用できる。
録音再生ボタンを押した際、演奏の繰り返しが検出されると自動的にDynamic Macroが適用され、
繰り返し部分が登録される。これにより、あるフレーズを最初から登録する意思がなくても、
演奏中や、演奏が終わってから登録できる。
また、1回半程度の繰り返しで検出できることから、
長いフレーズでも開始・終了位置を明確に意識することなく登録できる。

Dynamic Macroを活用することで、以下のような演奏が可能になる。

\paragraph*{ベースラインの上でインプロビゼーションする}
ああ
\paragraph*{イカしたフレーズに伴奏をつける (?)}
いい
\paragraph*{ほげほげする}
