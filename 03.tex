%3
\section{{\system}使用例}

\ref{repiano}で述べた単純な操作によって、
何気なく弾いたフレーズを重ねていったり、
試行錯誤をしながら合奏的演奏を楽しむことができる。

{\system}を利用したさまざまな演奏の楽しみ方について解説する。

\subsection{初心者でも上手く演奏できる例}
ピアノ初心者にとって両手を器用にコントロールするのは難しく、
伴奏とメロディラインを同時に演奏するのは大きなハードルになる。
{\system}では、両手が使えなくても、
片手で演奏できるフレーズを重ねていくことで完成度の高い演奏をすることが可能である。

またリズムやタイミングが正確でなくても、
やり直し機能で試行錯誤しながら演奏することができる。

% 速弾きができない人でもなんとかなる例(初心者?)
%  メロディや伴奏と同時に弾くのは無理でも、細かいフレーズをだけなら弾ける

\subsection{Dynamic Macroを利用する例}
Dynamic Macroとは、入力の繰り返しを自動化するEmacs拡張である。
連続した操作の開始と終了のタイミングを正確に指定する必要がなく、
実際の操作を行ったあとから繰り返し実行を指示できる。

{\system}では、演奏の登録時にこのDynamic Macroを適用できる。
録音再生ボタンを押した際、2回以上の演奏の繰り返しが検出されると自動的にDynamic Macroが適用され、
繰り返し部分が登録される。これにより、あるフレーズを最初から登録する意思がなくても、
演奏中や、演奏が終わってから登録できる。
また、長いフレーズでも1回半程度の繰り返しで検出できることから、開始・終了位置を明確に意識することなく登録できる。

Dynamic Macroを活用することで、以下のような演奏が可能になる。

\paragraph*{ベースラインの上で即興演奏する}
最初にベースラインやコードといった伴奏パートを録音しておき、
その上で自由にメロディラインを演奏すれば、
簡単にジャズ的な即興演奏を楽しむことができる。

\paragraph*{合奏へスムーズに移行する}
最初から演奏を登録したり合奏をはじめる意思がなくても、
演奏中に気に入ったフレーズを使ってすぐに合奏をはじめることができる。
そのフレーズを二度以上繰り返してから録音再生ボタンを押すと、
Dynamic Macroによって繰り返し部分のみが登録され、
スムーズに合奏に移行できる。

\subsection{作曲に活用する例}
楽器を触っていると、急に新しいフレーズを思いついてしまうことがある。
しかし、素早く記録することは難しく、
焦ってレコーダーを取り出そうとしているうちに忘れることが多い。
常に{\system}を使用していれば、
弾いた後からでも再生でき、忘れることを防げる。
伴奏もすぐに付けることができるので楽曲へのイメージをふくらませやすく、
作曲に活用できる。
