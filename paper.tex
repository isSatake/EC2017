%%
%% 研究報告用スイッチ(情報処理学会用ファイルをEC2017用に変更)
%% [techreq]
%%
%% 欧文表記無しのスイッチ(etitle,jkeyword,eabstract,ekeywordは任意)
%% [noauthor]
%%

\documentclass[submit,techreq]{ec2017}
%\documentclass[submit,techreq,noauthor]{ipsj}


\usepackage[dvips]{graphicx}
\usepackage{latexsym}

\def\Underline{\setbox0\hbox\bgroup\let\\\endUnderline}
\def\endUnderline{\vphantom{y}\egroup\smash{\underline{\box0}}\\}
\def\|{\verb|}

\setcounter{巻数}{57}%vol53=2012
\setcounter{号数}{10}
\setcounter{page}{1}

\long\def\system{Re:Piano}

\begin{document}

\title{{\system}: 演奏履歴を活用する楽器演奏支援システム}

\etitle{{\system}: Supporting music session with past performance}

\affiliate{政メ}{慶應義塾大学政策・メディア研究科\\
Keio University}
\affiliate{環境}{慶應義塾大学環境情報学部\\
Keio University}
  
\author{佐竹 紘明}{Hiroaki Satake}{政メ}[stk@sfc.keio.ac.jp]
\author{増井 俊之}{Toshiyuki Masui}{環境}[masui@pitecan.com]

\begin{abstract} % 結論を冒頭に書く
  
演奏履歴をリアルタイムに活用できる楽器及びそれを利用した新しい演奏法を提案する。
%
録音された演奏データが演奏後に加工されて活用されることは多いが、
演奏時に演奏データをリアルタイムに活用できるシステムや楽器はほとんど存在しない。
%
 楽器の演奏履歴を常に記録しておき、
演奏時にそれをリアルタイムに利用することで多彩な演奏を可能にするシステム
「{\system}」を試作した。
%
本論文では{\system}の思想、実装手法、及び利用経験について述べる。
  
\end{abstract}

%\begin{jkeyword}
%情報処理学会論文誌ジャーナル,\LaTeX,スタイルファイル,べからず集
%\end{jkeyword}

\begin{eabstract}

We propose a new musical instrument that supports realtime reuse of
past performance data.
%
Using our system, \textit{\system},
a player can easily ask the system to replay the repeated phrase or
previously played phrase just by pressing a ``repeat'' button after the performance.
%
By recording all the performance data and reusing it at any time,
both novice players and experienced players can enjoy playing musical instruments
by having ensamble with their past performance.
  
\end{eabstract}

%\begin{ekeyword}
%IPSJ Journal, \LaTeX, style files, ``Dos and Dont's'' list
%\end{ekeyword}

\maketitle

%
% 1
%
\section{はじめに}
\label{sec:start}

ひとりで楽器を演奏するのはそれほど楽しくないものである。
これは楽曲として聞こえる演奏をひとりで行なうのが難しいからである。
特に初心者の場合、 自分の下手な演奏を聞くのが悲しいし、満足できるほど上手くなるまで練習するのは大変である。
初心者でない場合でも、単音だけ聞いて楽しい曲はほとんど無いことから、単音楽器をひとりで楽しく演奏することは難しい。
楽器をひとりで演奏して楽しいのは、熟練者がピアノやギターといった独奏楽器を弾くときぐらいかもしれない。

一方、下手であっても合奏に参加するのは楽しいものである。
カラオケは歌が下手でも人数が多ければ盛り上がることができるし、
小学校でのリコーダーや鍵盤ハーモニカによる楽しい合奏は、技量に関係なく誰もが体験している。
演奏技術が足りなくても、合奏のように沢山の音を重ねて重厚な音楽を作れれば、誰もが楽器演奏を楽しむことが可能である。
初心者や単音楽器奏者でも、合奏的に演奏を楽しむことができれば練習や上達にも効果があると思われる。

その場限りで楽器演奏を行なうのではなく、
演奏履歴を利用して音を重ねたり、
連続再生したりして過去の自分と簡単に合奏することができれば、
初心者でも、単音楽器でももっと演奏を楽しめる可能性がある。

ひとりでも合奏的な演奏を可能にするツールは既に存在しているが、さまざまな制約が存在する。

DTM (Desktop Music)\footnote{パソコンと楽器やマイクなどを接続し、
    専用のソフトウェアで演奏したり音楽制作を行う行為}では、
ひとりで録音した演奏を重ねていく多重録音が
ポピュラーな手法としてよく行なわれている\cite{jacob}\cite{resound}。

% これがDTMの例というのはよくわからない (増井)
% DTMというより多重録音の例ですねぇ (satake)

また、サンプラー\footnote{音声を登録し、
    ボタンなどのインターフェースを利用してそれらを自由に再生できる装置}
やルーパー\footnote{音声を重ねて記録し、ループ再生する装置}
といったツールを利用すると音声素材や演奏を登録し、それらを自由に呼び出して演奏に活用したり、
繰り返し再生して伴奏をさせることが可能である。

これらのツールはあらかじめ演奏全体の完成図が見えており、
レコーディングやライブパフォーマンスといった高い完成度を求められる利用シーンでは威力を発揮するが、
明確な録音開始・停止の操作が必要であり、なおかつタイミングの正確さが求められるため、
日常的な楽器演奏において何気なく利用するにはハードルが高い。

誰もが日常的に合奏的な演奏を楽しむためにはこのような制約を解決し、
より簡単に演奏履歴を活用できるシステムが必要である。
本論文では、このような思想にもとづいて作成した{\system}を提案する。

%2
\section{\system}
\label{repiano}

%
%{\system}では常に演奏を記録しておくことで、演奏が終わった後からでも素材として登録することが可能である。
%また、Dynamic Macroを利用することで繰り返しフレーズの開始・終了位置を意識しなくても
%演奏を登録することが可能になっている。

{\system}は、自分の過去の演奏履歴を活用することによって演奏をより楽しくすることができるシステムである。
演奏中に以下のボタンを押すことで楽しい演奏ライフをサポートする。

%2.1
\subsection{録音再生ボタン}
\label{recplaybutton}
(図入れたい)

直前の無音部分から現在までの演奏を登録して、繰り返し再生を行う。
録音開始の操作は不要で、演奏の途中や、
演奏が終わってから登録可能な点が既存のツールとは異なる大きな特徴である。
登録部分の中に演奏の繰り返しが含まれる場合はDynamic Macro\cite{masui}を適用し、
繰り返し部分だけを登録して連続再生を行う。
また、再生中に重ねて演奏を行なうことができ、そこで録音再生ボタンを押すとその演奏も新たに登録される。
この演奏は最初に登録された繰り返しフレーズのタイミングに合わせて記録されるため、
時間が経過してもずれることなく再生され続ける。

%2.2
\subsection{やり直しボタン}
(図入れたい)

\ref{recplaybutton}で重ねていった演奏を、新しいものから順に取り消す。
(文章増やしたい)

%3
\section{{\system}使用例}

\ref{repiano}で述べた単純な操作によって、
何気なく弾いたフレーズを重ねていったり、
試行錯誤をしながら合奏的演奏を楽しむことができる。

{\system}を利用したさまざまな演奏の楽しみ方について解説する。

\subsection{初心者でも上手く演奏できる例}
ピアノ初心者にとって両手を器用にコントロールするのは難しく、
伴奏とメロディラインを同時に演奏するのは大きなハードルになる。
{\system}では、両手が使えなくても、
片手で演奏できるフレーズを重ねていくことで完成度の高い演奏をすることが可能である。

またリズムやタイミングが正確でなくても、
やり直し機能で試行錯誤しながら演奏することができる。

% 速弾きができない人でもなんとかなる例(初心者?)
%  メロディや伴奏と同時に弾くのは無理でも、細かいフレーズをだけなら弾ける

\subsection{Dynamic Macroを利用する例}
Dynamic Macroとは、入力の繰り返しを自動化するEmacs拡張である。
連続した操作の開始と終了のタイミングを正確に指定する必要がなく、
実際の操作を行ったあとから繰り返し実行を指示できる。

{\system}では、演奏の登録時にこのDynamic Macroを適用できる。
録音再生ボタンを押した際、演奏の繰り返しが検出されると自動的にDynamic Macroが適用され、
繰り返し部分が登録される。これにより、あるフレーズを最初から登録する意思がなくても、
演奏中や、演奏が終わってから登録できる。
また、1回半程度の繰り返しで検出できることから、
長いフレーズでも開始・終了位置を明確に意識することなく登録できる。

Dynamic Macroを活用することで、以下のような演奏が可能になる。

\paragraph*{ベースラインの上で即興演奏する}
最初にベースラインやコードといった伴奏パートを録音しておき、
その上で自由にメロディラインを演奏すれば、
簡単にジャズ的な即興演奏を楽しむことができる。

\paragraph*{合奏へスムーズに移行する}
最初から演奏を登録したり合奏をはじめる意思がなくても、
演奏中に気に入ったフレーズを使ってすぐに合奏をはじめることができる。
そのフレーズを1回半程度繰り返してから録音再生ボタンを押すと、
Dynamic Macroによって繰り返し部分のみが登録され、
スムーズに合奏に移行できる。

\subsection{作曲に活用する例}
楽器を触っていると、急に新しいフレーズを思いついてしまうことがある。
しかし、素早く記録することは難しく、
焦ってレコーダーを取り出そうとしているうちに忘れることが多い。
常に{\system}を使用していれば、
弾いた後からでも再生でき、忘れることを防げる。
伴奏もすぐに付けることができるので楽曲へのイメージをふくらませやすく、
作曲に活用できる。

%4
\section{実装}

\subsection{Web技術による実装}

{\system}はJavaScriptとWeb MIDI API\cite{webmidi}で実装されたWebブラウザアプリケーションであり、
MIDIキーボードを接続することで利用できる。
最新のWebブラウザでは、
Web MIDI APIを利用することで簡単にMIDI機器と連携したアプリケーションを開発することが可能であり、
本論文のような実験も容易である。
また、Web Audio API\cite{webaudio}を利用して音源をブラウザ内に埋め込んでおり、
外部MIDI音源を利用することなく演奏を楽しむことができる。

\subsection{MIDIデータへのDynamic Macroの適用}

%音声データだけでなく、インターバルの時間も含めて繰り返し判定していることを書く
%MIDIメッセージにはノート情報と、デルタタイムが存在する
%本家ではコマンド情報だけを見ていればよかったが、MIDIだと時間にも注目する必要がある
%和音を認識するために補正していたり、ある程度の誤差を許容していることを書く
%1音くらい抜けても補完してくれる機能があるといいと思ってる

%5
\section{議論}
%\subsection{繰り返し録音ボタンは演奏の後で押せるのがポイント}

%繰り返しフレーズを録音したい場合
%タマタマ良いフレーズが弾けたときそれを再生できるのは嬉しい
%普通の重ね録音も楽勝で可能

\subsection{既存の手法との比較}
サンプラーはどんな音声でも登録できることから、単一の楽器だけでなく、
バリエーション豊かな音源によって多彩な演奏を可能にする強力なツールである。
{\system}では他の楽器の演奏履歴を呼び出すことはできないが、
ひとりでも気軽に合奏的演奏を楽しめることを目標としており、
音声の登録といった事前の準備は必ずしも必要ではなく、
{\system}を用意するだけですぐに利用できる。

ルーパーは、演奏を繰り返し再生する利用形態が{\system}に似ている。
単純な仕組みでありながらひとりでも重厚な音楽を作り出す事が可能であり、利用者も多い。
また、サンプラーや各種エフェクタなど他の機材を柔軟に組み合わせることも可能である。
{\system}ではルーパーのように繰り返し再生して音を重ねていくだけでなく、
\begin{itemize}
\item 常に録音していて、演奏が終わってからでも登録できる
\item Dynamic Macroが利用できる
\end{itemize}
という特徴を持つ。
これによって、
\begin{itemize}
\item あるフレーズを最初から登録する意思がなくても、
  演奏中や演奏が終わってから繰り返し再生を開始し、
  すぐに合奏を始められる
\item 登録したいフレーズの開始・終了位置を意識すること無く繰り返し再生の操作ができる
\end{itemize}
というルーパーにはない演奏支援機能が実現された。


%TwkyrやSamplrといった新しいルーパーシステムにも言及する?
%Concept Tahoeもイケる?気軽にルーパーを使いたいという目的は近い


%カラオケとの比較
%バックが上手 vs バックが存在しない
% みんなで楽しい vs ぼっち


\subsection{評価}

著者らが{\system}の試作品を約2ヶ月に渡って使用し、
得られた結果および評価について述べる。

第一著者はクラシックギターに習熟しているので、
ピアノでもある程度楽譜を見て演奏したり、
耳コピ\footnote{\textsf{音楽を耳で聞いて演奏を再現したり、楽譜に起こすこと}}
をすることはできるが、両手を使った演奏をすることはできない。
単音による自身のピアノ演奏をいつも退屈だと感じていたが、
{\system}を日常的に使い始めてからは、そのような意識が緩和された。
繰り返し再生の特性上、技巧的なコードチェンジをしながら演奏を展開していくことは難しいが、
ベースラインやコードを記録してからメロディラインを自由に演奏するスタイルなら
1つ1つのパートに集中でき、なおかつリズムも乱れづらく初心者にとってハードルが低い。
これによって、孤独な日常の練習であっても手軽に合奏的演奏を楽しむことができた。
現状より少しでもレベルの高い演奏を行えるようになることで、
継続的な楽器演奏へのモチベーションが高まった。

第二著者はジャズピアノの演奏を趣味としている。
ピアノでジャズを演奏する場合、ベース/バッキング/メロディを同時に弾きれば嬉しいものであるが、
二本の手でこれらを演奏することは難しいためどこかに負担がかかってしまう。
{\system}を利用して最初にベースパートを録音しておいてから
両手でピアノを重ねて演奏することにより、
簡単にジャズ演奏を楽しめるようになった。

%6
\section{結論}

%〜手法を提案した。

2つのボタン操作で演奏履歴を活用し、
ひとりでも気軽に合奏的演奏を楽しめるシステム{\system}を提案した。

%著者らによる(ユーザーテスト)では、〜ことが確かめられた。
%〜だと考える。
今後は、MIDIを用いない生楽器でも{\system}のような演奏を実現する方法について検討していきたい。


% 誰もが日常的に合奏的な演奏を楽しむためにはこのような制約を解決し、
% より簡単に演奏履歴を活用できるシステムが必要である。


\bibliographystyle{ipsjsort}
\bibliography{paper}

\end{document}
