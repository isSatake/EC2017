%5
\section{議論}

本当にたーのしーのか?
Repeanoが適する/適さない利用シーン

\subsection{繰り返し録音ボタンは演奏の後で押せるのがポイント}

繰り返しフレーズを録音したい場合
タマタマ良いフレーズが弾けたときそれを再生できるのは嬉しい
普通の重ね録音も楽勝で可能

\subsection{既存の手法との比較}

% 既存の手法の問題点は1章で書く

%DTM
%\footnote{\textsf{DeskTop Music\\パソコンと電子楽器などを接続し、専用のソフトウェアで演奏したり音楽制作を行う行為}}
%では、ひとりで録音した演奏を重ねていく多重録音はポピュラーな手法としてよく行なわれている。(youtube)(resoundbottle)
%しかし、機材を用意したり、専用ソフトウェアの操作を覚えるのは大変である。
%楽曲としての完成度が求められる利用シーンには適するが、演奏を行いながらリアルタイムに演奏履歴を活用することは難しい。

%サンプラーを利用すると、音声素材や演奏を登録し、それらを自由に呼び出して演奏に利用できる。
%これは事前に素材を用意しておくことが前提となっており、それなりの準備が必要である。
%こういったことから、演奏中に偶然思いついたフレーズを記録するのも難しい。

%ルーパーを利用すると、演奏を繰り返し再生してひとりでも合奏的な演奏を楽しむことができる。
%しかし、明確な録音開始・停止の操作が必要であり、タイミングを合わせるのが大変である。
%また常に録音しておき、演奏に活用するという用途には利用できない。
%試行錯誤しづらい

%TwkyrやSamplrといった新しいルーパーシステムにも言及する
%Concept Tahoeもイケる?気軽にルーパーを使いたいという目的は近い

常に演奏を記録しておくことで、良いと感じたフレーズをすぐに再利用できるのは本システムの重要なポイントである。
良いフレーズを忘れてしまうことを防げるし、すぐ演奏に活用できることでより演奏を楽しむことができる。

そのときの演奏内容に応じて録音・再生する音声を簡単に変更できることで、{\system}の利用可能なシーンは大きく広がる。
セッションの雰囲気に合わせて再生する音声を変えたり

演奏履歴を活用する楽器演奏
孤独な日常の練習でも気軽に合奏的な演奏を楽しむことができる

カラオケとの比較
バックが上手 vs バックが存在しない
みんなで楽しい vs ぼっち

\subsection{評価}

著者らが{\system}の試作品を約2ヶ月に渡って使用し、
得られた結果および評価について述べる。

% なお、著者はピアノ初心者であり、クラシックギターの演奏経験がある。
% 増井先生はジャズピアノ、ギター、トランペットに習熟している。

% 使う前より楽しくなったと言いたい

第一著者はクラシックギターに習熟しているので、
ピアノでもある程度楽譜を見て演奏したり、
耳コピ\footnote{\textsf{音楽を耳で聞いて演奏を再現したり、楽譜に起こすこと}}
をすることはできるが、両手を使った演奏をすることはできない。
単音による自身のピアノ演奏をいつも退屈だと感じていたが、
{\system}を日常的に使い始めてからは、そのような意識が緩和された。
繰り返し再生の特性上、技巧的なコードチェンジをしながら演奏を展開していくことは難しいが、
ベースラインやコードを記録してからメロディラインを自由に演奏するスタイルなら
1つ1つのパートに集中でき、なおかつリズムも乱れづらく初心者にとってハードルが低い。
これによって、手軽に合奏的演奏を楽しむことができた。
現状より少しでもレベルの高い演奏を行えるようになることで、
継続的な楽器演奏へのモチベーションが高まった。

第二著者はジャズピアノの演奏を趣味としている。
ピアノでジャズを演奏する場合、ベース/バッキング/メロディを同時に弾きれば嬉しいものであるが、
二本の手でこれらを演奏することは難しいためどこかに負担がかかってしまう。
{\system}を利用して最初にベースパートを録音しておいてから
両手でピアノを重ねて演奏することにより、
簡単にジャズ演奏を楽しめるようになった。

% Dynamic Macroを使えば長い繰り返しフレーズをサボれるから良い
% 前より楽器を触る時間が増えた
% こんなに上達した
% etc
