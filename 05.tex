%5
\section{議論}

本当にたーのしーのか?
Repeanoが適する/適さない利用シーン

\subsection{繰り返し録音ボタンは演奏の後で押せるのがポイント}

繰り返しフレーズを録音したい場合
タマタマ良いフレーズが弾けたときそれを再生できるのは嬉しい
普通の重ね録音も楽勝で可能

\subsection{既存の手法との比較}
サンプラーはどんな音声でも登録できることから、単一の楽器だけでなく、
バリエーション豊かな音源によって多彩な演奏を可能にする強力なツールである。
{\system}では他の楽器の演奏履歴を呼び出すことはできないが、
ひとりでも気軽に合奏的演奏を楽しめることを目標としており、
音声の登録といった事前の準備は必ずしも必要ではなく、
{\system}を用意するだけですぐに利用できる。


\subsection{評価}

著者らが{\system}の試作品を約2ヶ月に渡って使用し、
得られた結果および評価について述べる。

第一著者はクラシックギターに習熟しているので、
ピアノでもある程度楽譜を見て演奏したり、
耳コピ\footnote{\textsf{音楽を耳で聞いて演奏を再現したり、楽譜に起こすこと}}
をすることはできるが、両手を使った演奏をすることはできない。
単音による自身のピアノ演奏をいつも退屈だと感じていたが、
{\system}を日常的に使い始めてからは、そのような意識が緩和された。
繰り返し再生の特性上、技巧的なコードチェンジをしながら演奏を展開していくことは難しいが、
ベースラインやコードを記録してからメロディラインを自由に演奏するスタイルなら
1つ1つのパートに集中でき、なおかつリズムも乱れづらく初心者にとってハードルが低い。
これによって、手軽に合奏的演奏を楽しむことができた。
現状より少しでもレベルの高い演奏を行えるようになることで、
継続的な楽器演奏へのモチベーションが高まった。

第二著者はジャズピアノの演奏を趣味としている。
ピアノでジャズを演奏する場合、ベース/バッキング/メロディを同時に弾きれば嬉しいものであるが、
二本の手でこれらを演奏することは難しいためどこかに負担がかかってしまう。
{\system}を利用して最初にベースパートを録音しておいてから
両手でピアノを重ねて演奏することにより、
簡単にジャズ演奏を楽しめるようになった。
