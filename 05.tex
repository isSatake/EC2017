%5
\section{議論}

本当にたーのしーのか?
Repeanoが適する/適さない利用シーン
\subsection{繰り返し録音ボタンは演奏の後で押せるのがポイント}
繰り返しフレーズを録音したい場合
タマタマ良いフレーズが弾けたときそれを再生できるのは嬉しい
普通の重ね録音も楽勝で可能

\subsection{既存の手法との比較}

%hogeをするのに、既存のfugaという手法よりも本論文の手法の方がこういう点で優れていたり、こういう制約がある、ということを述べる
%hoge -> 演奏履歴を活用する楽器演奏
%fuga -> DTMなどにおける多重録音、サンプラー、ルーパー

DTM
\footnote{\textsf{DeskTop Music\\パソコンと電子楽器などを接続し、専用のソフトウェアで演奏したり音楽制作を行う行為}}
では、ひとりで録音した演奏を重ねていく多重録音はポピュラーな手法としてよく行なわれている。(youtube)(resoundbottle)
しかし、機材を用意したり、専用ソフトウェアの操作を覚えるのは大変である。
楽曲としての完成度が求められる利用シーンには適するが、演奏を行いながらリアルタイムに演奏履歴を活用することは難しい。

サンプラーを利用すると、音声素材や演奏を登録し、それらを自由に呼び出して演奏に利用できる。
これは事前に素材を用意しておくことが前提となっており、それなりの準備が必要である。
こういったことから、演奏中に偶然思いついたフレーズを記録するのも難しい。

常に演奏を記録しておくことで、良いと感じたフレーズをすぐに再利用できるのは本システムの重要なポイントである。
良いフレーズを忘れてしまうことを防げるし、すぐ演奏に活用できることでより演奏を楽しむことができる。

そのときの演奏内容に応じて録音・再生する音声を簡単に変更できることで、Re:Pianoの利用可能なシーンは大きく広がる。
セッションの雰囲気に合わせて再生する音声を変えたり

単純な操作だから、試行錯誤しやすい

演奏履歴を活用する楽器演奏
孤独な日常の練習でも気軽に合奏的な演奏を楽しむことができる

%試行錯誤しづらい
%これまた専用の機材が必要
%でもRe:Pianoにもブラウザが必要→結局一緒でわ

ルーパーを利用すると、演奏を繰り返し再生してひとりでも合奏的な演奏を楽しむことができる。
しかし、明確な録音開始・停止の操作が必要であり、タイミングを合わせるのが大変である。
また常に録音しておき、演奏に活用するという用途には利用できない。
試行錯誤しづらい


Re:Piano->操作が単純なので試行錯誤しても苦にならない
Dynamic Macroが使えるので、正確に録音開始・停止位置を指定する必要がない

カラオケとの比較
バックが上手 vs バックが存在しない
みんなで楽しい vs ぼっち

\subsection{評価}

% 主観的な評価
% こんな曲が弾けるようになった
% 前より楽器を触る時間が増えた
% こんなに上達した
% etc
