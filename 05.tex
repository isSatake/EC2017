%5
\section{議論}
%\subsection{繰り返し録音ボタンは演奏の後で押せるのがポイント}

%繰り返しフレーズを録音したい場合
%タマタマ良いフレーズが弾けたときそれを再生できるのは嬉しい
%普通の重ね録音も楽勝で可能

\subsection{既存の手法との比較}
サンプラーはどんな音声でも登録できることから、単一の楽器だけでなく、
バリエーション豊かな音源によって多彩な演奏を可能にする強力なツールである。
{\system}では他の楽器の演奏履歴を呼び出すことはできないが、
ひとりでも気軽に合奏的演奏を楽しめることを目標としており、
音声の登録といった事前の準備は必ずしも必要ではなく、
{\system}を用意するだけですぐに利用できる。

ルーパーは、演奏を繰り返し再生する利用形態が{\system}に似ている。
単純な仕組みでありながらひとりでも重厚な音楽を作り出す事が可能であり、利用者も多い。
また、サンプラーや各種エフェクタなど他の機材を柔軟に組み合わせることも可能である。
{\system}ではルーパーのように繰り返し再生して音を重ねていくだけでなく、
\begin{itemize}
\item 常に録音していて、演奏が終わってからでも登録できる
\item Dynamic Macroが利用できる
\end{itemize}
という特徴を持つ。
これによって、
\begin{itemize}
\item あるフレーズを最初から登録する意思がなくても、
  演奏中や演奏が終わってから繰り返し再生を開始し、
  すぐに合奏を始められる
\item 登録したいフレーズの開始・終了位置を意識すること無く繰り返し再生の操作ができる
\end{itemize}
というルーパーにはない演奏支援機能が実現されている。

%TwkyrやSamplrといった新しいルーパーシステムにも言及する?
%Concept Tahoeもイケる?気軽にルーパーを使いたいという目的は近い


%カラオケとの比較
%バックが上手 vs バックが存在しない
% みんなで楽しい vs ぼっち


\subsection{評価}

著者らが{\system}の試作品を約2ヶ月に渡って使用し、
得られた結果および評価について述べる。

第一著者はクラシックギターに習熟しているので、
ピアノでもある程度楽譜を見て演奏したり、
耳コピ\footnote{\textsf{音楽を耳で聞いて演奏を再現したり、楽譜に起こすこと}}
をすることはできるが、両手を使った演奏をすることはできない。
単音による自身のピアノ演奏をいつも退屈だと感じていたが、
{\system}を日常的に使い始めてからは、そのような意識が緩和された。
繰り返し再生の特性上、技巧的なコードチェンジをしながら演奏を展開していくことは難しいが、
ベースラインやコードを記録してからメロディラインを自由に演奏するスタイルなら
1つ1つのパートに集中でき、なおかつリズムも乱れづらく初心者にとってハードルが低い。
これによって、孤独な日常の練習であっても手軽に合奏的演奏を楽しむことができた。
現状より少しでもレベルの高い演奏を行えるようになることで、
継続的な楽器演奏へのモチベーションが高まった。

第二著者はジャズピアノの演奏を趣味としている。
ピアノでジャズを演奏する場合、ベース/バッキング/メロディを同時に弾ければ嬉しいものであるが、
二本の手でこれらを演奏することは難しいためどこかに負担がかかってしまう。
{\system}を利用して最初にベースパートを録音しておいてから
両手でピアノを重ねて演奏することにより、
簡単にジャズ演奏を楽しめるようになった。
