%4
\section{実装}

\subsection{Web技術による実装}

{\system}はJavaScriptとWeb MIDI API\cite{webmidi}で実装されたWebブラウザアプリケーションであり、
MIDIキーボードを接続することで利用できる。
最新のWebブラウザでは、
Web MIDI APIを利用することで簡単にMIDI機器と連携したアプリケーションを開発することが可能であり、
本論文のような実験も容易である。
また、Web Audio API\cite{webaudio}を利用して音源をブラウザ内に埋め込んでおり、
外部MIDI音源を利用することなく演奏を楽しむことができる。
現在、Google Chromeに対応した試験運用版をhttps://stkay.github.io/RePiano/で利用できる。

\subsection{楽器演奏へのDynamic Macroの適用}
本来、Dynamic Macroはテキストエディタのコマンドに適用するものであり、
入力されるコマンド情報のみで繰り返し検出を行う。
これを楽器演奏にも適用するためには、入力される音情報だけでなく、音同士の間隔も記録する必要がある。
特にMIDIでは、この間隔のことをデルタタイムという。

{\system}では、MIDIイベントのタイムスタンプからデルタタイムを計算し、
繰り返し検出に利用している。
このとき、50ms以内の誤差は許容している。
また和音を演奏する際は、同時に弾いたつもりでもMIDIイベントが前後して記録されてしまうことがある。
これを正確に繰り返すのは不可能なので、15ms以内のデルタタイムを全て0msに補正している。

% 登録したいフレーズが演奏されたとき、
% 元々鳴ってたフレーズとの間隔(デルタタイム)を保持しておくことで、
% ズレずに繰り返し再生できる

%1音くらい抜けても補完してくれる機能があるといいと思ってる
