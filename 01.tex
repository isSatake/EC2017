%
% 1
%
\section{はじめに}
\label{sec:start}

ひとりで楽器を演奏するのはそれほど楽しくないものである。
これは楽曲として聞こえる演奏をひとりで行なうのが難しいからである。
特に初心者の場合、 自分の下手な演奏を聞くのが悲しいし、満足できるほど上手くなるまで練習するのは大変である。
初心者でない場合でも、単音だけ聞いて楽しい曲はほとんど無いことから、単音楽器をひとりで楽しく演奏することは難しい。
楽器をひとりで演奏して楽しいのは、熟練者がピアノやギターといった独奏楽器を弾くときぐらいかもしれない。

一方、下手であっても合奏に参加するのは楽しいものである。
カラオケは歌が下手でも人数が多ければ盛り上がることができるし、
小学校でのリコーダーや鍵盤ハーモニカによる楽しい合奏は、技量に関係なく誰もが体験している。
演奏技術が足りなくても、合奏のように沢山の音を重ねて重厚な音楽を作れれば、誰もが楽器演奏を楽しむことが可能である。
初心者や単音楽器奏者でも、合奏的に演奏を楽しむことができれば練習や上達にも効果があると思われる。

その場限りで楽器演奏を行なうのではなく、
演奏履歴を利用して音を重ねたり、
連続再生したりして過去の自分と簡単に合奏することができれば、
初心者でも、単音楽器でももっと演奏を楽しめる可能性がある。

ひとりでも合奏的な演奏を可能にするツールは既に存在しているが、さまざまな制約が存在する。

DTM\footnote{\textsf{DeskTop Music}: パソコンと楽器やマイクなどを接続し、
    専用のソフトウェアで演奏したり音楽制作を行う行為}
  では、ひとりで録音した演奏を重ねていく多重録音は
  ポピュラーな手法としてよく行なわれている。\cite{jacob}\cite{resound}

  % これがDTMの例というのはよくわからない (増井)
  % DTMというより多重録音の例ですねぇ (satake)

また、サンプラー\footnote{\textsf{音声を登録し、
    ボタンなどのインターフェースを利用してそれらを自由に再生できる装置}}
やルーパー\footnote{\textsf{音声を重ねて記録し、ループ再生する装置}}
といったツールを利用すると音声素材や演奏を登録し、それらを自由に呼び出して演奏に活用したり、
繰り返し再生して伴奏をさせることが可能である。

これらのツールはあらかじめ演奏全体の完成図が見えており、
レコーディングやライブパフォーマンスといった高い完成度を求められる利用シーンでは威力を発揮するが、
明確な録音開始・停止の操作が必要であり、なおかつタイミングの正確さが求められるため、
日常的な楽器演奏において何気なく利用するにはハードルが高い。

誰もが日常的に合奏的な演奏を楽しむためにはこのような制約を解決し、
より簡単に演奏履歴を活用できるシステムが必要である。
本論文では、このような思想にもとづいて作成した{\system}を提案する。
