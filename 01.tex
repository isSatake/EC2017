%1
\section{はじめに}
\label{sec:start}

ひとりで楽器を演奏するのはそれほど楽しくないものである。
これは楽曲として聞こえる演奏をひとりで行なうのが難しいからである。
特に初心者の場合、 自分の下手な演奏を聞くのが悲しいし、満足できるほど上手くなるまで練習するのは大変である。
初心者でない場合でも、単音だけ聞いて楽しい曲はほとんど無いことから、単音楽器をひとりで楽しく演奏することは難しい。
楽器をひとりで演奏して楽しいのは、熟練者がピアノやギターといった独奏楽器を弾くときぐらいかもしれない。

一方、下手であっても合奏に参加するのは楽しいものである。
カラオケは歌が下手でも人数が多ければ盛り上がることができるし、小学校でのリコーダーや鍵盤ハーモニカによる楽しい合奏は、技量に関係なく誰もが体験している。
演奏技術が足りなくても、合奏のように沢山の音を重ねて重厚な音楽を作れれば、誰もが楽器演奏を楽しむことが可能である。
初心者や単音楽器奏者でも、合奏的に演奏を楽しむことができれば練習や上達にも効果があると思われる。

その場限りで楽器演奏を行なうのではなく、演奏履歴を利用して音を重ねたり、連続再生したりして過去の自分と簡単に合奏することができれば、
初心者でも、単音楽器でももっと演奏を楽しめる可能性がある。
本論文では、こういう思想にもとづいて作成した{\system}を提案する。

% ルーパー、サンプラー等のダメな点を述べる
% 既存の機材は、難しい
% 解決策は2章以降で述べる

ひとりでも合奏的な演奏を可能にする機材は既に存在しているが、さまざまな制約が存在する。

ルーパーを利用すると、録音した演奏を繰り返し再生してパートを増やしていったり、伴奏をさせることが可能である。
あらかじめ演奏全体の完成図が見えており、
レコーディングやライブパフォーマンスといった高い完成度を求められる利用シーンでは威力を発揮するが、
明確な録音・開始の操作が必要であり、なおかつタイミングの正確さが求められるため、
日常的な楽器演奏において何気なく利用するにはハードルが高い。

<サンプラー、DTMの話>

本論文ではこのような制約を解決し、日常的に、何気なく演奏履歴を活用して楽器演奏を楽しむことができるシステムRe:Pianoを提案する。

%
%Re:Pianoでは常に演奏を記録しておくことで、演奏が終わった後からでも素材として登録することが可能である。
%また、Dynamic Macroを利用することで繰り返しフレーズの開始・終了位置を意識しなくても
%演奏を登録することが可能になっている。
